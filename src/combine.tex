\documentclass{article}
\usepackage[margin=1.2cm]{geometry}
\usepackage{graphicx}

\begin{document}
\begin{figure}
    \begin{minipage}{0.91\textwidth}
    \includegraphics[width=0.48\textwidth]{GERES/castor_geres.pdf}
    \includegraphics[width=0.48\textwidth]{GERES/castor_geres_integrated.pdf}
    %\includegraphics[width=0.48\textwidth]{TXAR/columbia_txar.pdf}
    \end{minipage}
    \begin{minipage}{0.91\textwidth}
		    %\includegraphics[width=0.48\textwidth]{PDAR/columbia_pdar.pdf}
    \centering
    %\includegraphics[width=0.48\textwidth]{YKA/columbia_yka.pdf}
    \caption{Event: Castor, 1\textsuperscript{st} of Octover, 2013. Magnitude 4.3. Traces show velocity (left) and integrated velocity (right) normalized by their maximum amplitude. Band pass filter: 0.7-5 Hz. Theoretical P and pP arrivals are plotted for different source depths (blue traces), while the observed array beam (black trace)
    has been shifted to the depth with best visual fit. The measured backazimuth and slowness given in seconds per km of the beams are indicated in (brackets).}
    \end{minipage}
\end{figure}
\begin{figure}
    \begin{minipage}{0.61\textwidth}
    \includegraphics[width=10cm]{map.pdf}
    \end{minipage}
    \begin{minipage}{0.30\textwidth}
    \includegraphics[width=0.99\textwidth]{velocity_model_source.pdf}
    \includegraphics[width=0.99\textwidth]{velocity_models.pdf}
    \end{minipage}
    \caption{Array and event locations, as well as the underlying Crust 2.0 velocity models. The sources site receiver model is covered with a 100 m water layer. Epicentral distance [degrees]: GERES: 12.6. }
\end{figure}

\end{document}
